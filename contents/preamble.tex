%% fonty
\usepackage{fontspec}
\setmainfont[Ligatures=TeX]{Stempel Garamond}
\defaultfontfeatures{Mapping=tex-text}
\usepackage{textcase}
\usepackage{booktabs}
\sloppy
\renewcommand{\textsc}[1]{\MakeTextUppercase{#1}} % příjmení autorů velkými písmeny

%% typografická zlepšení
%\usepackage{csquotes}

% barvy
\usepackage{xcolor}
\definecolor{mygreen}{cmyk}{0.36,0.0,0.47,0.40}

\usepackage[
   bookmarks=true,
   unicode=true,
   pdftitle={Zaměstnanecká díla na školách},
   pdfauthor={Jakub Michálek},
   pdfkeywords={otevřený přístup} {zaměstnanecké dílo} {vzdělávání},
   colorlinks=false,
   linkbordercolor=mygreen,% 
   citebordercolor=mygreen,%   
   urlbordercolor=mygreen%
   ]{hyperref}

\let\oldurl\url
\renewcommand{\url}[1]{<\oldurl{#1}>}

% nadpisy
\usepackage{titlesec}
\titleformat{\section}
{\Large\bfseries}
{\thesection}{.5em}{}[\titleline{\color{mygreen}\titlerule[3pt]}] 

\titleformat{\subsection}
{\large\bfseries}
{\thesubsection}{.5em}{}[\titleline{\color{mygreen}\titlerule[1pt]}] 

\titleformat{\paragraph}[display]
{\normalfont\bfseries}{Čl.~\arabic{paragraph}}{1em}{}

\let\oldmarginpar\marginpar%
\renewcommand{\marginpar}[1]{%
  \oldmarginpar[\raggedleft #1]{\raggedright #1}%
}

%% floats get barriers
\usepackage[section]{placeins}

%% poznámky pod čarou
\usepackage{footnote} % možnost používat poznámky pod čarou v tabulkách
\makesavenoteenv{tabular} % automaticky se budou ukládat poznámky pod čarou v prostředí tabular
\makesavenoteenv{table} % automaticky se budou ukládat poznámky pod čarou v prostředí table
% zůstává problém, že při kliknutí na odkaz na poznámku pod čarou vytvořený pomocí hyperref nefunguje

% hlavičky
\usepackage{fancyhdr}
\pagestyle{fancy}
%\renewcommand{\chaptermark}[1]{\rightmark{\thechapter\ #1}}
%\renewcommand{\sectionmark}[1]{\rightmark{\thesection\ #1}}

% Nastaví styl záhlaví pro sudé i liché stránky
\fancyhf{} % smaže aktuální nastavení záhlaví a zápatí
\fancyfoot[LE]{\bfseries\thepage} 
\fancyfoot[RO]{\bfseries\thepage} 
\fancyhead[LO]{\itshape\rightmark}
\fancyhead[RE]{\itshape\leftmark}

\renewcommand{\headrulewidth}{0pt} % tloušťka linky
\renewcommand{\footrulewidth}{0pt}   % patička chybí
\addtolength{\headheight}{1.2pt} % prostor pro záhlaví

\fancypagestyle{plain}{
  \fancyhead{} % na prázdných stránkách nechci záhlaví
  \renewcommand{\headrulewidth}{0pt} % ani linku
}


%% citování
\usepackage[backend=biber,
style=footnote-dw,%authortitle-dw
namefont=smallcaps,
isbn=true,
language=english,
autocite=footnote,
backref=true,
hyperref,
nopublisher=false]{biblatex} 
\urlstyle{rm}

\DeclareFieldFormat{title}{\mkbibemph{#1}} % titul kurzívou
\let\cite\autocite

\defbibheading{bibliography}{\section*{Seznam použité literatury}} % přejmenování sekce

%% závěrečné úkony
\AtEndDocument{
\addcontentsline{toc}{section}{Seznam použité literatury}
\printbibliography} % tisk bibliografie na konci souboru

\let\finalandcomma=\!
\renewcommand*{\multinamedelim}{\addcomma\space}
\renewcommand*{\finalnamedelim}{%
\ifnum\value{liststop}>2 \finalandcomma\fi%
\addsemicolon\space}
\renewcommand*{\bibmultinamedelim}{\addcomma\space}
\renewcommand*{\bibfinalnamedelim}{%
\ifnum\value{liststop}>2 \finalandcomma\fi%
\addsemicolon\space}%
\renewcommand*{\citemultinamedelim}{\addsemicolon\space}
\renewcommand*{\citefinalnamedelim}{\addsemicolon\space}
\renewcommand*{\labelnamepunct}{\addperiod\space}
\renewcommand*{\nametitledelim}{\addperiod\space}
\renewcommand*{\newunitpunct}{\addperiod\space}

\DefineBibliographyStrings{english}{%
seenote          = {viz pozn\adddot},
page          = {s\adddot},
pages          = {s\adddot},
backrefpage = {citováno na s\adddot},
backrefpages= {citováno na s\adddot},
url = {Dostupný z WWW\addcolon},
byeditor         = {ed\adddotspace},
urlseen         = {cit\adddot},
}

% popiseky tabulek
\usepackage[hang,bf,small]{caption} % úprava popisku tabulky
\setlength{\captionmargin}{20pt}

% tisk částí
\let\stdsection\section
\renewcommand*{\section}{\clearpage\stdsection}

\AtEveryCite{%
\clearfield{url}%
\clearfield{urldate}%
 \DeclareFieldFormat{url}{}%
 \DeclareFieldFormat{urldate}{}%
}

% Formát data v url
\DeclareFieldFormat{urldate}{%
  \iffieldundef{urlday}
    {}
    {\stripzeros{\thefield{urlday}}\adddot%
\stripzeros{\thefield{urlmonth}}\adddot%
\printfield{urlyear}}%
}

\renewbibmacro{url+urldate}{%
\iffieldundef{urlday}
    {}
  {\space\printtext{[}\bibstring{urlseen} \printurldate\printtext{]}} 
\iffieldundef{url}
{}
{\printfield{url}\adddotspace}}

\DeclareFieldFormat{url}{\bibstring{url} \url{#1}}

% Správné zpětné citace
\renewcommand*{\finentrypunct}{}
\renewbibmacro*{pageref}{%
  \addperiod% NEW
  \iflistundef{pageref}
    {}
    {\printtext[parens]{% NEW
       \ifnumgreater{\value{pageref}}{1}
         {\bibstring{backrefpages}\ppspace}
     {\bibstring{backrefpage}\ppspace}%
       \printlist[pageref][-\value{listtotal}]{pageref}\addperiod}}}% NEW
